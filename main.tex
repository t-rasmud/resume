%% start of file `template.tex'.
%% Copyright 2006-2013 Xavier Danaux (xdanaux@gmail.com).
%
% This work may be distributed and/or modified under the
% conditions of the LaTeX Project Public License version 1.3c,
% available at http://www.latex-project.org/lppl/.


\documentclass[11pt,a4paper,sans]{moderncv}        % possible options include font size ('10pt', '11pt' and '12pt'), paper size ('a4paper', 'letterpaper', 'a5paper', 'legalpaper', 'executivepaper' and 'landscape') and font family ('sans' and 'roman')

% moderncv themes
\moderncvstyle{casual}                             % style options are 'casual' (default), 'classic', 'oldstyle' and 'banking'
\moderncvcolor{blue}                               % color options 'blue' (default), 'orange', 'green', 'red', 'purple', 'grey' and 'black'
%\renewcommand{\familydefault}{\sfdefault}         % to set the default font; use '\sfdefault' for the default sans serif font, '\rmdefault' for the default roman one, or any tex font name
%\nopagenumbers{}                                  % uncomment to suppress automatic page numbering for CVs longer than one page

% character encoding
\usepackage[utf8]{inputenc}                       % if you are not using xelatex ou lualatex, replace by the encoding you are using
%\usepackage{CJKutf8}                              % if you need to use CJK to typeset your resume in Chinese, Japanese or Korean

% adjust the page margins
\usepackage[scale=0.75]{geometry}
%\setlength{\hintscolumnwidth}{3cm}                % if you want to change the width of the column with the dates
%\setlength{\makecvtitlenamewidth}{10cm}           % for the 'classic' style, if you want to force the width allocated to your name and avoid line breaks. be careful though, the length is normally calculated to avoid any overlap with your personal info; use this at your own typographical risks...

% personal data
\name{\huge{}}{\huge{Rashmi Mudduluru}}
\title{\large{\\rashmi4@cs.washington.edu  
\\ PhD student
\\ CSE2 380, Paul G. Allen School of Computer Science \& Engineering}}
\begin{document}
%\begin{CJK*}{UTF8}{gbsn}                          % to typeset your resume in Chinese using CJK
%-----       resume       ---------------------------------------------------------
\makecvtitle

\section{Research Interests}
\cvitem{}{My research interests lie broadly in the areas of Software Engineering, Programming Languages and Verification.}

\section{Education}
\cventry{}{PhD Candidate, Computer Science and Engineering}{University of Washington}{Seattle (2017 - )}
{}{CGPA: 3.6/4, Advisor: Prof. Michael Ernst} 
\cventry{}{M.Sc(Engg), Computer Science}{Indian Institute of Science}{India}
{}{Received 2015, CGPA: 6.3/8, Advisor: Dr. Murali Krishna Ramanathan \newline{} Thesis: Efficient Instrumentation for Object Flow Profiling}  % arguments 3 to 6 can be left empty
%\cventry{2006--2010}{B.E, Computer Science}{Anna University}{India}{}{CGPA - 9.24/10}

\section{Professional Experience}
\cventry{}{Research Fellow}{Microsoft Research, India}{Bangalore}{2015--2017}{Mentor: Dr. Akash Lal}
\cventry{}{Research Intern}{Facebook}{Seattle}{June 2019--September 2019}{Mentors: Dr. Herman Venter, Dr. Shaz Qadeer}
\cventry{}{Research Intern}{Amazon}{Seattle}{June 2020--September 2020}{Mentors: Prof. Michael Ernst}

\section{Research Projects}
\cvlistitem{\textbf{Research at UW}: \textit{Compile-time verification of correctness properties related to collections}:
    \begin{enumerate}
        \item Built a type system that can express and verify determinism properties of sequential programs.
        The key ideas in the type system are type qualifiers for nondeterminism, order-nondeterminism, and determinism; type well-formedness rules to restrict the typings for collections; and enhancements to polymorphism that improve precision when analyzing collection operations. We implemented our type system for Java. Our type checker warns if a program is nondeterministic or verifies that the program is deterministic. We formalized the type system and proved its soundness. In case studies of 90097 lines of code, the Determinism Checker found 87 previously-unknown nondeterminism errors, even in programs that had been heavily vetted by developers who were greatly concerned about nondeterminism errors.
        \item My current research projects aims to verify iteration correctness over collections. Iterating over empty collections leads to run-time errors. It is not uncommon for these bugs to escape into production. Our type-system based solution has qualifiers that express whether the \textit{hasNext} property of an Iterator is true or false and that can express whether the Collection
        is empty or not. Extant approaches look at this as an API misuse issue and propose heuristic-based  solutions and therefore, cannot guarantee soundness. In a comparative study, our tool reported all the bugs found by previous techniques with high precision. As ours is a sound approach, our tool did not report a few of the false positives reported by the heuristic-based tools.  
    \end{enumerate}}
   
\cvlistitem{\textbf{Research at Facebook}: Worked on formally proving the safety of LibraBFT consensus protocol. This protocol
    ensures that nodes in a blockchain system have a consistent view of the blockchain. 
    We formalized the critical safety invariants that should be maintained by the implementation in order to guarantee the safety properties
ensured by the protocol and added these invariants as a set of machine-verifiable asserts to the LibraBFT codebase.}
\cvlistitem{\textbf{Research at Amazon}: Built a tool to verify the integrity of 
    user data stored in the cloud. Even data that is protected by cryptographic hashes is prone to hardware errors threatening the safety contracts expected by the customers. We identified program locations that must have consistent data to ensure integrity. We built a compile-time checker that tracks the cryptographic hashes corresponding to user data and verifies that they are consistent a these locations.}
    
\cvlistitem{\textbf{Research at MSR}: \begin{enumerate}
        \item Created a runtime analysis that identifies liveness bugs in distributed systems. The analysis detected repeating states in a given execution trace that violated a liveness property. 
        Several features of the tool were designed with the aim of keeping the runtime overheads at a minimum.
        The tool identified real bugs in production code like Microsoft Azure storage vNext System. 
        \item Implemented a dynamic data race detector for programs written in P\#, a DSL for developing and testing asynchronous systems
        using the happens-before algorithm. I tracked all runtime memory reads and writes with the help of Extended Reflection, a library that facilitates dynamic instrumentation by inserting callbacks for MSIL instructions.
\end{enumerate} }

\cvlistitem{\textbf{Masters thesis}:Designed and built a profiler that tracks the precise data path taken by objects in Java programs. I implemented
this tool in Java on top of the Calfuzzer/Soot framework. This tool helped identify performance issues caused by memory bloat 
in several Java programs. Fixing these issues significantly improved the performance of the programs under test.}
%\cvlistitem{Built an incremental static analyzer on top of SATURN, a constraint based
%static analysis tool for finding bugs. Implementation was done in OCaml and Berkeley DB.}

\section{Conference Publications}
\cvlistitem{Rashmi Mudduluru, Jason Waataja, Suzanne Millstein, Michael D. Ernst \textbf{Verifying Determinism in Sequential Programs}, International Conference on Software Engineering (ICSE), 2021.}
\cvlistitem{Rashmi Mudduluru, Pantazis Deligiannis, Ankush Desai, Akash Lal, Shaz Qadeer \textbf{Lasso detection using partial-state caching}, Formal Methods in
    Computer-Aided Design (FMCAD), 2017.}
\cvlistitem{Rashmi Mudduluru, Murali Krishna Ramanathan, \textbf{Efficient Flow Profiling for Detecting Performance Bugs}, 25th International Symposium on Software Testing and Analysis (ISSTA), 2016. \newline{}
    [Received the \textbf{ACM SIGSOFT Distinguished paper award}].}
\cvlistitem{Pantazis Deligiannis, Matt McCutchen, Paul Thomson, Shuo Chen, Alastair F. Donaldson, 
    John Erickson, Cheng Huang, Akash Lal, Rashmi Mudduluru, Shaz Qadeer, Wolfram Schulte, \textbf{Uncovering Bugs in Distributed Storage Systems during Testing (Not in Production!)}, 14th USENIX Conference on File and Storage Technologies (FAST), 2016.}
\cvlistitem{Monika Dhok, Rashmi Mudduluru, Murali Krishna Ramanathan, \textbf{Pegasus: Automatic Barrier Inference for Stable Multithreaded Systems}, International Symposium on Software Testing and Analysis (ISSTA), 2015.}
\cvlistitem{Rashmi Mudduluru, Murali Krishna Ramanathan, \textbf{Efficient Incremental Static Analysis Using Path Abstraction}, 17th International Conference on Fundamental Approaches to Software Engineering (FASE), 2014.}

\section{Graduate Course Projects}
\cvlistitem{Built an interactive data visualization tool in d3 that summarizes the relationship between pairs
of genome structures as part of a three member team.}
\cvlistitem{Built a code search tool that matches a given input query on the parse tree of the source code.
This was done in a team of three.}
\cvlistitem{Implemented a static null pointer dereference analysis tool for Java programs using the
Soot bytecode analysis framework.}

%\section{Graduate Courses}
%\cvitem{}{Concepts of Programming Languages, Software Engineering, Data Visualization, Program analysis and verification, 
%Computer Systems, Design and analysis of algorithms, Automata theory and
%computability, Topics in software bug detection, Automated verification.}

\section{Awards and Scholarships}
\cvlistitem{ACM SIGSOFT Distinguished paper award for the paper \textit{Efficient Flow Profiling for Detecting Performance Bugs} at ISSTA 2016.}
\cvlistitem{Received the PLMW scholarship for attending POPL 2015.}
%\cvlistitem{Secured 99.7 percentile in Graduate Aptitude Test in Engineering (GATE) 2012.}

\section{Service}
\cvlistitem{Teaching Assistant for the under graduate course \textit{Data Structures and Algorithms} at the University of Washington.}
\cvlistitem{Teaching Assistant for the under graduate course \textit{Software Engineering} at the University of Washington.}
\cvlistitem{Teaching Assistant for the graduate course \textit{Automata Theory and Computability} at the Indian Institute of Science.}
%\cvlistitem{Student volunteer at \textit{International Conference on Software Engineering (ICSE) 2014.}}
\cvlistitem{Served in the Artifact Evaluation Committees of PLDI 2017, ISSTA 2017, and ISSTA 2018}

%\section{References}
%\cvitem{}{Dr. Akash Lal \newline{} Senior Researcher, \newline{} Microsoft Research, India,
%\newline{} \href{https://www.microsoft.com/en-us/research/people/akashl/}{https://www.microsoft.com/en-us/research/people/akashl/}\newline{}}

%\cvitem{}{Dr. Murali Krishna Ramanathan \newline{} Assistant Professor, Department of Computer Science and 
%Automation, \newline{} Indian Institute of Science,
%\newline{} \href{http://drona.csa.iisc.ac.in/~muralikrishna/}{http://drona.csa.iisc.ac.in/\textasciitilde muralikrishna/}\newline{}}

%\cvitem{}{Dr. Deepak D'Souza \newline{} Associate Professor, Department of Computer Science and 
%Automation, \newline{} Indian Institute of Science,
%\newline{} \href{http://drona.csa.iisc.ernet.in/~deepakd/}{http://drona.csa.iisc.ernet.in/\textasciitilde deepakd/}}

\end{document}
